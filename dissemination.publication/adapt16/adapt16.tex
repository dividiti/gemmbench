% http://www.acm.org/sigs/publications/proceedings-templates
\documentclass{acm_proc_article-sp} % Tested with v3.2SP (April 2009)

\usepackage[hyphens]{url}
\usepackage{booktabs}

\begin{document}

\title{{GEMMbench}: a framework for reproducible and collaborative benchmarking
of matrix multiplication} 
%
\subtitle{\LARGE \url{https://github.com/dividiti/gemmbench}}

\numberofauthors{1}
\author{
\alignauthor
  Anton Lokhmotov, {\Large \tt dividiti}\\
  \email{\Large \url{anton@dividiti.com}}\\
  \affaddr{ideaSpace West}\\
  \affaddr{3 Charles Babbage Road}\\
  \affaddr{Cambridge, CB3 0GT}\\
  \affaddr{United Kingdom}\\
}

\maketitle

\begin{abstract}

The generic matrix-matrix multiplication (GEMM) is arguably the most popular
computational kernel of the 20th century. 
%
Yet, surprisingly, no common methodology for evaluating GEMM performance 
has been established over the many decades of using GEMM for comparing
architectures, compilers and ninja-class programmers.

We introduce GEMMbench, a framework and methodology for evaluating performance
of GEMM implementations.
%
GEMMbench is implemented on top of Collective Knowledge (CK), a lightweight
framework for reproducible and collaborative R\&D in computer systems.
%
Using CK allows the R\&D community to crowdsource hand-written and
compiler-generated GEMM implementations and to study their performance across
multiple platforms, data sizes and data types.
%
Our initial implementation supports hand-written OpenCL kernels operating on
matrices consisting of single- and double-precision floating-point values, and
producing single or multiple output elements per work-item (via thread
coarsening and vectorization).


\end{abstract}

% A category with the (minimum) three required fields
% \category{H.4}{Information Systems Applications}{Miscellaneous}
%A category including the fourth, optional field follows...
% \category{D.2.8}{Software Engineering}{Metrics}[complexity measures, performance measures]

\terms{Software}

%\keywords{ACM proceedings, \LaTeX, text tagging}

\section{Introduction}

The generic matrix-matrix multiplication (GEMM) is given by the equation:
%
\begin{displaymath} C = \alpha A \times B + \beta C \end{displaymath}
%
\noindent where $A$, $B$ and $C$ are matrices, and $\alpha$ and $\beta$ are
scalars.

GEMM is arguably the most popular computational kernel of the 20th century.
%
The apparent simplicity of GEMM has haunted generation after generation of
researchers who have evaluated its performance on generation after generation
of computer systems,\footnote{Conveniently, both for researchers and computer
systems a generation means 3--4 years.} while uncovering layer after layer of
its hidden complexity.
%
For example, discovering the beneficial effects of cache blocking on GEMM
performance~\cite{Lam:1991} has fuelled research on locality optimizations in
compilers for many years.

%
Yet, surprisingly, no common methodology for evaluating GEMM performance has
been established over the many decades of using this kernel for comparing
architectures, compilers and ninja-class programmers.
%
Consequently, the reader of a report presenting GEMM results is often left wondering:

\begin{itemize}
%
\item Was the kernel specialized, for example, to $C = A \times B$? (In other
words, $\alpha=1$ and $\beta=0$.) 
%
\item Which of the data types were used: single precision (SGEMM), double
precision (DGEMM), complex single precision (CGEMM), or complex double
precision (ZGEMM)?
%
\item Which data layouts were used: normal (N) or transposed (T)?\footnote{For
matrices stored in row-major order, $C = \alpha A \times B^{T} + \beta C$
typically results in better locality, because $B^{T}$ is read row-wise.} If
transposed, did the execution time include the overhead for transposition?
%
\item Which data shapes were used: square or rectangular? If rectangular, did
the execution time depend on the ratio between the dimensions? 
%
\item Which data sizes were used: small or large?
%
\item On a system with caches, did `large' result in cache thrashing; did
`small' result in good locality (no thrashing)?
%
\item On a heterogeneous system equipped with a discrete accelerator, did the
execution time include the overhead for copying the data to the accelerator and
back, or only the kernel execution time?
%
\item Did the evaluation include power or energy measurements?
%
\item If a diesel generator was used to get the system running, how many
megaflops per gallon were they
getting?\footnote{\url{http://www.hpcwire.com/2006/06/30/the_new_limits_on_high_performance_computing-1/}}
%
\item More seriously, have we achieved significant improvements in energy
efficiency of floating-point operations over the last decade?\footnote{For
DGEMM, ClearSpeed's CSX600 processor provided 25 Gflops/s in under 10 Watts.}
%
\item How much human effort and ingenuity was involved in writing the kernel or
in implementing the compiler that generated the kernel?
%
\item Can we compare the generators, for example, based on polyhedral
compilation~\cite{Beaugnon:2014} and functional expression
rewriting~\cite{Steuwer:2015} in a fair way (including code quality, code generation
time and robustness)?
%
\item Can we evaluate the generators against ninja-class
programmers~\cite{Goto:2008} or vendor libraries?
%
\item Have we used all the tricks up our sleeves to get the fastest
GEMM implementation for our hardware and problem at hand?
%
\item Can we {\em adapt} our GEMM implementations to work well across a range of
architectures, data types, data sizes, etc.?
%
\end{itemize}

Given that are discussing something apparently as simple as GEMM, intended to
give us insights for solving more complex `real-world' problems, it is
essential to start getting some of the answers right to facilitate our learning
and knowledge sharing.

We introduce GEMMbench, a framework and methodology for evaluating performance
of GEMM implementations.
%
GEMMbench is implemented on top of Collective Knowledge (CK), a lightweight
framework for reproducible and collaborative R\&D in computer systems.%
\footnote{\url{http://cknowledge.org}}
%
Our initial implementation supports hand-written OpenCL kernels operating on
matrices consisting of single- and double-precision floating-point values, and
producing single or multiple output elements per work-item (via thread
coarsening and vectorization).
%
Over time, we plan to involve the community to add further hand-written and
generated kernels (e.g. from \cite{Beaugnon:2014,Steuwer:2015}), and,
importantly, to collectively study the GEMM performance across multiple
platforms, data sizes and data types.


\section{Details}

%
The GEMMbench framework reads from a JSON file the metadata describing a
kernel.
%
The JSON file specifies the data type ({\tt S} or {\tt D}), the layout of the
matrices ({\tt N} or {\tt T}), the thread-coarsening configuration ({\tt di}
for the number of rows and {\tt dj} for the number of columns in a block
computed by a single work-item), and so on.

For example, the SGEMM kernel that assumes that $A$ is non-transposed and $B$
is transposed and outputs a single element per work-item: 
%
\begin{verbatim}
kernel void gemm(
    global float const * restrict A,
    global float const * restrict B,
    global float       * restrict C,
    float alpha, float beta, uint n)
{
    const uint j = get_global_id(0);
    const uint i = get_global_id(1);

    float ABij = 0.0f;
    for (uint k = 0; k < n; k += 1)
    {
        ABij += A[i*n + k] * B[j*n + k];
    }
    C[i*n + j] = alpha * ABij + beta * C[i*n + j];
}
\end{verbatim}
%
is described by the following metadata:
%
\begin{verbatim}
{
    "name"   : "SGEMM_NT_1x1",
    "file"   : "SGEMM_NT_1x1.cl",
    "type"   : "S",
    "transA" : "N",
    "transB" : "T",
    "dj"     : 1,
    "di"     : 1
}
\end{verbatim}

See further examples in the {\tt dataset} entries of the GEMMbench repository.%
\footnote{\url{https://github.com/dividiti/gemmbench/tree/master/dataset}}


\section{How to reproduce?}

Briefly, install Collective Knowledge\footnote{\url{http://github.com/ctuning/ck}}
and follow the steps:
%
\begin{verbatim}
$ ck pull repo:gemmbench \
    --url=https://github.com/dividiti/gemmbench
$ cd `ck find repo:gemmbench`
$ ck compile
$ ck run
\end{verbatim}
%
More information will appear on the GEMMbench page during the period of public discussions.

Table~\ref{SGEMM_NT:table}.

\begin{table*}
\centering
\caption{\label{SGEMM_NT:table}The execution time of 3 SGEMM NT kernels on an Odroid-XU3 board. 
Chip: Samsung Exynos~5422. GPU: ARM Mali-T628 (4 cores). GPU frequency: 600 MHz. OpenCL driver: v4.0.}
\begin{tabular}{l|l|l|rrrr}
\toprule
{\bf OpenCL program} & {\bf Local work size} & {\bf Matrix order}      &       0 &       1 &       2 &       3 \\
\midrule
SGEMM\_NT\_1x1.cl & (8, 8) & 64   &         &         &         &         \\
                &        & 96   &   2.947 &   2.958 &   2.974 &   2.839 \\
                &        & 128  &   2.701 &   2.683 &   2.666 &   2.686 \\
                &        & 192  &   3.057 &   3.036 &   3.036 &   3.072 \\
                &        & 256  &   2.922 &   3.002 &   2.943 &   2.947 \\
                &        & 384  &   2.735 &   2.846 &   2.874 &   2.820 \\
                &        & 512  &   2.884 &   2.883 &   2.892 &   2.917 \\
                &        & 640  &   2.704 &   2.617 &   2.643 &   2.579 \\
                &        & 768  &   2.795 &   2.766 &   2.768 &   2.795 \\
                &        & 896  &   2.816 &   2.823 &   2.797 &   2.819 \\
                &        & 1024 &   2.789 &   2.770 &   2.784 &   2.787 \\
\midrule
SGEMM\_NT\_4x1.cl &        & 64   &   2.803 &   2.838 &   2.781 &   2.832 \\
                &        & 96   &  10.188 &  10.384 &  10.348 &  10.263 \\
                &        & 128  &  10.670 &  10.728 &  10.505 &  10.677 \\
                &        & 192  &  12.220 &  12.091 &  11.770 &  12.032 \\
                &        & 256  &  11.870 &  11.584 &  11.220 &  11.545 \\
                &        & 384  &  10.620 &   3.868 &   3.521 &   3.471 \\
                &        & 512  &  10.506 &  10.395 &  10.413 &  10.396 \\
                &        & 640  &   1.980 &   1.876 &   1.868 &   1.944 \\
                &        & 768  &  10.281 &  10.322 &  10.258 &   9.657 \\
                &        & 896  &   2.115 &   2.484 &   2.025 &   1.938 \\
                &        & 1024 &  10.366 &   7.994 &  10.316 &   2.350 \\
\midrule
SGEMM\_NT\_4x1\_barrier.cl &        & 64   &   1.869 &   1.749 &   2.502 &   1.583 \\
                &        & 96   &   5.781 &   5.808 &   5.774 &   5.802 \\
                &        & 128  &   7.819 &   7.865 &   7.768 &   7.818 \\
                &        & 192  &  11.054 &  11.698 &  11.589 &  11.703 \\
                &        & 256  &   8.672 &   8.718 &   8.405 &   8.516 \\
                &        & 384  &   8.745 &   9.636 &   9.502 &   9.168 \\
                &        & 512  &  10.037 &   9.993 &  10.011 &  10.102 \\
                &        & 640  &   9.140 &   9.409 &   9.326 &   9.328 \\
                &        & 768  &   9.798 &   9.871 &   9.831 &   9.857 \\
                &        & 896  &   9.922 &   9.911 &   9.925 &   9.859 \\
                &        & 1024 &   9.827 &   9.813 &   9.799 &   9.831 \\
\bottomrule
\end{tabular}

\end{table*}

\section*{Acknowledgments}

We thank Grigori Fursin, CTO of {\tt dividiti} and Chief Scientist of the
cTuning foundation, for designing and implementing the Collective
Knowledge framework, on top of which we implemented GEMMbench.

\bibliographystyle{abbrv}
\bibliography{adapt16}


%\balancecolumns

\end{document}
